% Magda Gregorova, 2025-01-10

\chapter{Introduction}\label{ch:Intro}
This \LaTeX\ template provides a basic structure plus some tips about writing your master's thesis manuscript. 
You shall not to keep the structure as is here, you shall change it as appropriate for your thesis.
You can even change the \LaTeX\ formatting and the whole thesis style if you wish.
The title-page shall, nevertheless, contain the information as provided in this template.
In particular, you shall decide:
\begin{itemize}[noitemsep, nosep]
\item on the structure of the thesis, the chapters and their content
\item inclusion of list of figures, list of tables, index, etc.
\item reasonable math notation and stick to it throughout the thesis
\end{itemize}

\section{Organization of template}\label{sec:Intro_orga}
This template make great use of the \verb|\input| commands to input files containing parts of the text.
This is generally recommended for longer texts as it helps you to organize your work better and allows for easier re-usability, e.g. of the preamble.
You can also easily change the order of the chapters or comment out complete chapters if not relevant.
Nevertheless, you do not have to stick to this practice if you do not find it convenient for your work.

Though the template provides some examples for standard formatting and for including classical elements into your text, such as figures or tables, it expects sufficient knowledge of \LaTeX\ from your side.
There is plenty of material about \TeX\ and \LaTeX\ available on line.
A classical book is for example \cite{oetikerNotShortIntroduction2023}.
There are many more things you can do to make your text look professionally and spending some time on polishing it so is strongly recommended.
An example of advanced tool you may want to explore on your own is the \href{https://ctan.org/pkg/pgf?lang=en}{PGF/TikZ} package for creating graphics.

For longer and more complicated documents it is recommended to install LaTex locally on your computer and use some of the dedicated LaTex editors such as Texstudio or standard code editors and their extensions such as VS Code and Latex Workshop. 
In-browser editing via tools such as Overleaf is discouraged as it does not provide sufficient flexibility for organizing your manuscript files.
Moreover, the free version does not have any versioning capability which is critical for keeping in control of your writing.
Remember to back-up and version your writing for example through GitHub or the faculty Bitbucket to be sure not to accidentally lose your work.

\section{Writing thesis}\label{sec:Intro_writing}
When writing the thesis manuscript keep in mind that it is the most important part of your master thesis work.
It is ``the'' document that will be read and evaluated and will remain even after you have left the school.
It shall explain well what, why and how have you done.

Follow the basic principles of good writing. 
Always keep the reader in the center of your attention and make sure that your text is well structured, easy to follow and possible to comprehend even for someone who has not been involved in developing the thesis.
Focus on the main points and do not get distracted by minor technical details.
If some detail might be useful or interesting, you can include them into the the appendix.

Remember that in scientific texts all claims have to be supported by evidence.
Such evidence can come from some previous work you cite, such as a text book, e.g. \cite{bishopPatternRecognitionMachine2006},
or it shall be directly provided by you, e.g. by results of experiments, mathematical proofs, etc.
There are many more good practices for writing which I am not going to explain here.