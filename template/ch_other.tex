% Magda Gregorova, 2025-01-10

\chapter{Literature review}\label{ch:Lit_review}

You shall review the state of the art relevant for your thesis. 
You may dedicate a chapter to it or include it somewhere where you discuss your method.
When discussing literature you shall always make clear how does it relate to your work:
Do you build on it and use some parts of it?
Do you improve on them and od something else?
Do you take inspiration and develop their ideas further?
Etc?

\chapter{Method}\label{ch:Method}

Here you describe what you actually did.
What method you used, how and why?
Give sufficient detail and be technically correct and precise.
This is where you may need to use quite a bit of math to make clear what you are doing.
You shall also use appropriate imagery (diagrams, etc.) to help the reader.
Remember a picture is worth a thousand words.
Spend some time thinking about how to support your text through suitable illustrations.

\chapter{Results}\label{ch:Results}

This chapter together with chapter \ref{ch:Method} about the method is the most important of your thesis.
Here you shall present the results of your experiments. 
The results shall be summarized in suitable tables and graphs. 
Make sure these are well formatted, easy to understand and read. 
Highlight results which are important in tables, use clear colors in graphs, make annotation sufficiently big to be readable. 

The presentation of your results, however, does not finish by summarizing them in tables and graphs.
Equally if not even more important is the interpretation of the results presented in tables and graphs.
Guide the reader through these by explaining what to look at, where to focus on, what to take away from it.
You shall never present a table or a graph without a reason - explain the reasons clearly.

Students often underestimate the importance of explaining the results in the tables and graphs. 
This is one of the most important steps of presenting results and shows that you understand the results well.

\chapter{Conclusions}\label{ch:Conclusions}

In the last chapter you shall conclude with summarizing the main points of the thesis and what has been achieved. 
You shall take a bit of step back here and look at the thesis more objectively.
Be open about the limitations of your work and take these as a starting point for possible future work which you shall outline.

\section{Personal takeaways}\label{sec:Personal}

You shall include a short section about personal takeaways from writing the thesis. 
This is typically not included in other types of scientific texts but a master thesis is rather specific and you are still learning. 
The hope therefore is that you will have learned something through developing the thesis. 
These might be technical skills as well as more soft skills such as time planning etc. 
You may also mention here what from your point of view went well and what less well and what you take away as lessons learned. 



